\documentclass[]{article}
\usepackage{listings}
\usepackage{csquotes}
\usepackage{xcolor}
\usepackage[linesnumbered,algoruled,boxed,commentsnumbered]{algorithm2e}
\begin{document}

\section{Quotation an Citation (4 marks)}
\subsection{Quotation (2 marks)}
The margins of the quotation environment are indented on both the left and the
right. The text is justified at both margins. Leaving a blank line between text
produces a new paragraph. The package \textbf{csquotes} offers a multilingual solu-
tion to quotations, with integration to citation mechanisms offered by BibTeX.
This package allows one for example to switch languages and quotation styles
according to babel language selections.

\begin{quote}
"Unlike the quote environment, each paragraph is indented nor- mally.
It’s important to remark that even if you are typing quotes on English
there are different quotation marks used in English (UK) and English
(US)."
\end{quote}

\subsection{Citation (2 marks)}
Latex [1] is a document preparation system for typesetting program. It is used
to create different types of document structures. A Latex file (.tex) is created
using any text editor (vim, emacs, gedit, etc.). There are also many LaTeX
IDEs like Kile, TexStudio, etc.. The Latex code is then compiled which creates
a standard (.pdf) file. Thus, the presentation of the document does not change
on different machines.
\\
Type style[2] is used to indicate logical structure. Emphasized text appears in
italic style type and input in typewriter style. Type style is specified by three
components: shape, series, and family.\\

There are two ways of producing a bibliography[3]. You can either produce
a bibliography by manually listing the entries of the bibliography or producing
it automatically using the BibTeX program of LaTeX. The bibliography style can
be declared with bibliographystyle command, which may be issued anywhere
after the preamble. The style is a file with .bst extension that determines how
bibliography entries will appear at the output, such as if they are sorted or not,
or how they are labeled etc. The extension .bib is not written explicitly. There
are many standard bibliography style files. Two of them that are compatible
with IIT thesis manual are plain.bst and alpha.bst. They are part of the LaTeX
package; a student does not need to download it. The plain.bst and alpha.bst
styles are explained below. The symbols in a math formula fall into different
classes that correspond more or less to the part of speech each symbol would
have if the formula were ex pressed in words. Certain spacing and position-
ing cues are traditionally used for the different symbol classes to increase the
readability of formulas. [4]

My citations are in proper order as per references ref1, ref2, ref3, and ref4.

\section{Algorithm and Pseudo Code (22 marks)}
\subsection{Listing (10 marks)}
\rule{\textwidth}{0.1pt}
\definecolor{codegreen}{rgb}{0,0.6,0}
\definecolor{codegray}{rgb}{0.5,0.5,0.5}
\definecolor{codepurple}{rgb}{0.58,0,0.82}
\definecolor{backcolour}{rgb}{1,1,1}

\lstdefinestyle{mystyle}{
    backgroundcolor=\color{backcolour},   
    commentstyle=\color{codegreen},
    keywordstyle=\color{blue},
    numberstyle=\tiny\color{backcolour},
    stringstyle=\color{codepurple},
    basicstyle=\large,
    breakatwhitespace=false,         
    breaklines=true,                 
    captionpos=b,                    
    keepspaces=true,                 
    numbers=left,                    
    numbersep=5pt,                  
    showspaces=false,                
    showstringspaces=false,
    showtabs=false,                  
    tabsize=2
}
\lstset{style=mystyle}

\begin{lstlisting}[language=C++]
//Breadth First Search Function
void BFS(list<long long int> queue,long long int length
	){
	long long int v ;
	if (queue.empty())
		return ;
	list<long long int>::iterator i;
	list<long long int> queue_temp;
	while(!queue.empty()){
		v=queue.front() ;
		queue.pop_front();
		for(i=adj[v].begin();i!=adj[v].end();i++){
			if(!pro_ver[*i]){
				result[*i]=length;
				queue_temp.push_back(*i);
				pro_ver[*i]=true;
				adj[*i].remove(v);
				}
			}
		}
	BFS(queue_temp , length+1);
}
\end{lstlisting}
\newpage
\subsection{Algorithmic (12 marks)}
\RestyleAlgo{boxruled}
\begin{algorithm}[H]
\caption{How to write algorithms}
\SetAlgoLined
\SetKwInput{KwData}{Input}
\KwData{A graph Graph and a starting vertex root of Graph}
\SetKwInput{KwResult}{Output}
\KwResult{All vertices’s reachable from root labeled as explored.}
Breadth-First-Search(Graph, root):

\For{each node n in Graph  :}{
$n$.\textbf{distance} = INFINITY

$n$.\textbf{parent} = NIL
}
create empty \textbf{queue} Q

root.\textbf{distance} = 0

Q.\textbf{enqueue}(root)

\While{Q is not empty :}{
current = Q.dequeue()

\For{each node n that is adj acent to current}{
\If{n.\textbf{distance} == INFINITY}{
n.\textbf{distance} = current.\textbf{distance} + 1 n.\textbf{parent} = current

Q.\textbf{enqueue}(n)
}
}
}
\end{algorithm}
\end{document}